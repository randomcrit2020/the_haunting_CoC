\section{Roxbury Sanitarium}

El Sanatorio de Roxbury fue construido a principios de siglo bajo el nombre de
“Hospital de Tisícos de Boston”, un lugar destinado a que los más pobres de la
ciudad pudieran recuperarse—o al menos morir con dignidad—de la tuberculosis.
Con el tiempo, sin embargo, las prioridades cambiaron. La línea tenue entre la
enfermedad física y el deterioro mental se desdibujó, y cada vez más recursos
del hospital se destinaron al tratamiento de distintas formas de locura.
Finalmente, el sanatorio fue reconvertido en un asilo, uno enteramente dedicado
a contener, tratar y, en muchos casos, simplemente custodiar a los desquiciados
de Boston.

Al entrar, lo primero que ves es un escritorio de recepción, tras el cual una
joven con uniforme blanco inmaculado escribe en un cuaderno de tapas cosidas.
Su caligrafía es meticulosa, su expresión neutra, como si los horrores
cotidianos que la rodean ya no la afectaran.

Una puerta conduce a un pasillo largo, y al caminar por él comienzas a sentir
el peso del lugar. A cada lado, figuras quebradas por la mente se deslizan en
silencio o balbucean en bucles infinitos. Algunos pacientes están sentados en
sillas de ruedas, moviendo las manos en el aire como si intentaran atrapar
cosas que solo ellos pueden ver. Otros vagan tambaleantes, los ojos perdidos,
como ecos de personas que una vez fueron.

\subsection{Versos bíblicos}

\begin{itemize}

    \item \emph{Hay una clase de hombre cuyos dientes son como espadas, y sus
    mandíbulas como cuchillos, hechos para devorar a los afligidos de la faz de
    la tierra. Por su propio instrumento es vencida la sanguijuela.}

    \item \emph{Sed sobrios, velad; porque vuestro adversario el diablo, como
    león rugiente, anda alrededor buscando a quien devorar.}

    \item \emph{Bienaventurado será el que tomare y estrellare a tus pequeñitos
    contra las piedras.}

    \item \emph{Deja que los muertos entierren a sus muertos.}

    \item \emph{No os unáis en yugo desigual con los incrédulos; porque ¿qué
    compañerismo tiene la justicia con la injusticia? ¿Y qué comunión la luz
    con las tinieblas?}

\end{itemize}