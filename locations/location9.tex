\section{The Old Corbitt Place}

Aquí se alza una casa, solitaria, con una arquitectura Federal envejecida que
recuerda los primeros suspiros de la nación. Es una reliquia de otra época,
sentada con obstinación en medio de un vecindario ahora dominado por lo
comercial, como si el tiempo la hubiese rodeado, pero no logrado desalojar.
Taciturna y ensimismada, permanece en penumbra bajo la sombra de estructuras
más nuevas y arrogantes que la flanquean.

La fachada gris verdosa mira hacia la calle con una quietud que incomoda. En la
parte trasera, un jardín descuidado se ahoga bajo maleza espesa, y un emparrado
medio derrumbado apunta al abandono progresivo. Al contemplar la casa, puedes
evitar notar cómo parece replegarse sobre sí misma, deslizándose hacia las
sombras como algo que prefiere no ser visto. Las ventanas, ocultas tras
cortinas pesadas, parecen aferrarse a sus secretos con una fuerza casi física,
como si temieran que, al ceder un poco, algo muy antiguo y muy oscuro pudiera
escapar.
