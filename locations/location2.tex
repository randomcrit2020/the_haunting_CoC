\section{Boston Globe} 

El Boston Globe, uno de los periódicos más antiguos y de mayor renombre en la
ciudad, con la mayor circulación de toda Nueva Inglaterra, late con el pulso
incesante de las noticias en marcha. El pequeño vestíbulo de entrada está
apenas amueblado y rebosa de movimiento: hombres y mujeres cruzan de un lado a
otro con prisa, llevando papeles, carpetas o simplemente la urgencia en el
rostro.

Más allá, se abre la redacción: una maraña de escritorios de madera golpeada y
archivadores maltrechos, todos tan apretujados que dos personas espalda con
espalda no podrían levantarse sin tropezar. La atmósfera es sofocante, cargada
de humo, tensión y la energía de algo que no se detiene. El ruido es constante:
teclas de máquinas de escribir que repiquetean como metralla, y voces que se
cruzan a gritos, discutiendo titulares, cerrando ediciones o pidiendo
confirmaciones.

No es un lugar cómodo, ni tranquilo, pero aquí, entre el desorden y la presión,
se forjan las historias que llenan las páginas del día siguiente, algunas
hostorias verídicas, otras no tanto, pero todas necesarias.

\subsection{The Morgue}

Una bombilla desnuda cuelga del techo, su luz amarillenta apenas alcanza a
rozar los rincones de esta habitación sombría, proyectando sombras largas y
extrañas sobre las hileras interminables de archivadores que llenan el espacio.
El calor es denso, casi sólido, atrapando en su interior el olor rancio del
moho y de tinta antigua. No es exactamente un depósito, piensas, sino algo más
fúnebre—un mausoleo, quizás—no de cuerpos, sino de historias olvidadas. Cada
gabinete cerrado parece guardar no solo papeles, sino fragmentos de vidas que
el tiempo ha dejado atrás, esperando en silencio a que alguien, cualquiera,
vuelva a abrirlas.
