\section{Boston Globe} 

El Boston Globe, uno de los periódicos más antiguos y de
mayor renombre en la ciudad, con la mayor circulación de toda Nueva Inglaterra,
late con el pulso incesante de las noticias en marcha. El pequeño vestíbulo de
entrada está apenas amueblado y rebosa de movimiento: hombres y mujeres cruzan
de un lado a otro con prisa, llevando papeles, carpetas o simplemente la
urgencia en el rostro.

Más allá, se abre la redacción: una maraña de escritorios de madera golpeada y
archivadores maltrechos, todos tan apretujados que dos personas espalda con
espalda no podrían levantarse sin tropezar. La atmósfera es sofocante, cargada
de humo, tensión y la energía de algo que no se detiene. El ruido es constante:
teléfonos que suenan sin cesar, teclas de máquinas de escribir que repiquetean
como metralla, y voces que se cruzan a gritos, discutiendo titulares, cerrando
ediciones o pidiendo confirmaciones.

No es un lugar cómodo, ni tranquilo, pero aquí, entre el desorden y la presión,
se forjan las historias que llenan las páginas del día siguiente—algunas
verídicas, otras no tanto, pero todas necesarias.
